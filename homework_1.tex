\documentclass[11pt]{article}

\input{./utils/useful_packages}
\input{./utils/short_notation}

\usepackage[nomessages]{fp}

\usepackage{listings}
\lstset{language=python,
    basicstyle=\ttfamily,
    keywordstyle=\bfseries,
	frame = single,
	numbers=right,
	tabsize=2,
}
\usepackage[letterpaper,margin=1in]{geometry}
\usepackage{fancyhdr}
\usepackage[explicit]{titlesec}
%\graphicspath{{./figures/homework_1}} 
%\usepackage{cmbright}
%\renewcommand{\familydefault}{\sfdefault}


\title{PROBLEM SET \#1}

\date{\today}

\author{APC 523/MAE 507/AST 523 : Numerical Algorithms for Scientific Computing \\ Vivek Kumar}

\hypersetup{
pdftitle= {\@title},
pdfauthor = {\@author},
pdfsubject = {},
pdfkeywords = {},
pdfmoddate= {\@date},
pdfcreator = {\@author},
pdftoolbar=true,        
pdfmenubar=true
}
\newcommand{\py}[1]{{\ttfamily #1}}



\makeatletter
\def\@maketitle{
\begingroup
\centering
{
{\Large \MakeUppercase{\@title}\par}
\vskip 0.1\baselineskip
{\normalsize\noindent\@author\par}
\vskip 0.1\baselineskip
{\normalsize\noindent \today \par}
}
\endgroup
}
\makeatother

\titleformat{\section}{\Large}{\MakeUppercase{}\thesection\quad}{0.1em}{{#1}}

\begin{document}

\maketitle

\section{Error in (symmetric) rounding vs chopping}
\textbf{Assertion}: When mapping a real number $x$ to a nearby machine number in $\Rbb(p,q)$, the upper bound in the relative error for symmetric rounding is:
	\begin{align*}
		\left|\frac{x-\mt{rd}(x)}{x}\right|\leq 2^{-p}
	\end{align*}
\textbf{Proof:}\\
Consider the number $x$ to be represented as:
	\begin{align*}
		x = \pm \left(\sum_{l=1}^{\infty} b_{-l}2^{-l}\right)2^e
	\end{align*}
If the number is to be rounded to $p$ terms, two cases arise:
	\begin{itemize}
	\item[] \textbf{CASE I.} The $(p+1)^{\mt{th}}$ is 0.\\
	In this scenario the difference between the true value and the rounded value is given by:
			\begin{align*}
				x - \mt{rd}(x) = \pm\left(\sum_{l=p+2}^{\infty} b_{-l}2^{-l}\right)2^e
			\end{align*}
			The maximum relative error can then be computed as:
			\begin{align*}
				\mt{max}\left|\frac{x-\mt{rd}(x)}{x}\right| 	&= \frac{\mt{max}\left|x-\mt{rd}(x)\right|}{\mt{min}\left|x\right|}\\
																&= \frac{2^{-p-1}2^e}{2^{-1}2^e }\\
																&= 2^{-p}
			\end{align*}
			which is what we set to prove.
	\item[] \textbf{CASE II.} The $(p+1)^{\mt{th}}$ is 1.\\
	In this scenario the maximum difference between the true and the rounded value is obtained as:
			\begin{align*}
				\mt{max}|x - \mt{rd}(x)| = \left(2^{-p} - 2^{-p-1}\right)2^e
			\end{align*}
			This is the case we all the leading terms from $(p+2)$ are $1$. Hence the maximum relative error can be computed as before:
			\begin{align*}
				\mt{max}\left|\frac{x-\mt{rd}(x)}{x}\right| 	&= \frac{\mt{max}\left|x-\mt{rd}(x)\right|}{\mt{min}\left|x\right|}\\
																&= \frac{\left(2^{-p} -2^{-p-1}\right)2^e}{2^{-1}2^e }\\
																&= 2^{-p}
			\end{align*}
			which is what we set to prove
	\end{itemize}
	Both the cases show that the maximum symmetric rounding off error is $2^{-p}$

\newpage
\section{An accurate implementation of $e^x$}
\begin{enumerate}[(a)]
	\item Work out the terms of the infinite series upto $n=30$ by rounding upto 5-significant figures\\
			\begin{table}[H]
			\centering
				\begin{tabular}{c|c|c|c}
					$n$ & numerator & denominator & $n^{\mt{th}}$ term \\
					\hline
					0	&	1.0000	&	1.0000	& 1.0000\\
					1 & 5.50000E+00 & 1.00000E+00 & 5.50000E+00\\
					2 & 3.02500E+01 & 2.00000E+00 & 1.51250E+01 \\
					3 & 1.66380E+02 & 6.00000E+00 & 2.77300E+01\\
					4 & 9.15090E+02 & 2.40000E+01 & 3.81290E+01\\
					5 & 5.03300E+03 & 1.20000E+02 & 4.19420E+01\\
					6 & 2.76820E+04 & 7.20000E+02 & 3.84470E+01\\
					7 & 1.52250E+05 & 5.04000E+03 & 3.02080E+01\\
					8 & 8.37380E+05 & 4.03200E+04 & 2.07680E+01\\
					9 & 4.60560E+06 & 3.62880E+05 & 1.26920E+01\\
					10 & 2.53310E+07 & 3.62880E+06 & 6.98050E+00\\
					11 & 1.39320E+08 & 3.99170E+07 & 3.49020E+00\\
					12 & 7.66260E+08 & 4.79000E+08 & 1.59970E+00\\
					13 & 4.21440E+09 & 6.22700E+09 & 6.76790E-01\\
					14 & 2.31790E+10 & 8.71780E+10 & 2.65880E-01\\
					15 & 1.27480E+11 & 1.30770E+12 & 9.74840E-02\\
					16 & 7.01140E+11 & 2.09230E+13 & 3.35100E-02\\
					17 & 3.85630E+12 & 3.55690E+14 & 1.08420E-02\\
					18 & 2.12100E+13 & 6.40240E+15 & 3.31280E-03\\
					19 & 1.16660E+14 & 1.21650E+17 & 9.58980E-04\\
					20 & 6.41630E+14 & 2.43300E+18 & 2.63720E-04\\
					21 & 3.52900E+15 & 5.10930E+19 & 6.90700E-05\\
					22 & 1.94100E+16 & 1.12400E+21 & 1.72690E-05\\
					23 & 1.06760E+17 & 2.58520E+22 & 4.12970E-06\\
					24 & 5.87180E+17 & 6.20450E+23 & 9.46380E-07\\
					25 & 3.22950E+18 & 1.55110E+25 & 2.08210E-07\\
					26 & 1.77620E+19 & 4.03290E+26 & 4.40430E-08\\
					27 & 9.76910E+19 & 1.08890E+28 & 8.97150E-09\\
					28 & 5.37300E+20 & 3.04890E+29 & 1.76230E-09\\
					29 & 2.95510E+21 & 8.84180E+30 & 3.34220E-10 \\
					30 & 1.62530E+22 & 2.65250E+32 & 6.12740E-11 \\
				\end{tabular}
			\end{table}
	\item Compute the $e^{5.5}$ using partial sums for left to right
			\begin{table}[H]
			\centering
				\begin{tabular}{c|c}
					$n$ & numerator & denominator & $n^{\mt{th}}$ term & value \\
					\hline
					0 & 1.0000\\
					1 & 6.50000E+00 \\
					2 & 2.16250E+01 \\
					3 & 4.93550E+01 \\
					4 & 8.74840E+01 \\
					5 & 1.29430E+02 \\
					6 & 1.67880E+02 \\
					7 & 1.98090E+02 \\
					8 & 2.18860E+02 \\
					9 & 2.31550E+02 \\
					10 & 2.38530E+02 \\
					11 & 2.42020E+02 \\
					12 & 2.43620E+02 \\
					13 & 2.44300E+02 \\
					14 & 2.44570E+02 \\
					15 & 2.44670E+02 \\
					16 & 2.44700E+02 \\
					17 & 2.44710E+02 \\
					18 & 2.44710E+02 \\
					19 & 2.44710E+02 \\
					20 & 2.44710E+02 \\
					21 & 2.44710E+02 \\
					22 & 2.44710E+02 \\
					23 & 2.44710E+02 \\
					24 & 2.44710E+02 \\
					25 & 2.44710E+02 \\
					26 & 2.44710E+02 \\
					27 & 2.44710E+02 \\
					28 & 2.44710E+02 \\
					29 & 2.44710E+02 \\
					30 & 2.44710E+02 \\
				\end{tabular}
			\end{table}
			\begin{list}
				\item For $k=17$, $e^{5.5}$ converges to 5-significant figures
				\item The value of $e^{5.5}$ computed using built-in function is $244.691932$
				\item The relative error is $0.000074$
			\end{list}
	\item For summing right to left
			\begin{table}[H]
			\centering
				\begin{tabular}{c|c}
					$n$ & value \\
					\hline
					0 & 3.34220E-10 \\
					1 & 2.09650E-09 \\
					2 & 1.10680E-08 \\
					3 & 5.51110E-08 \\
					4 & 2.63320E-07 \\
					5 & 1.20970E-06 \\
					6 & 5.33940E-06 \\
					7 & 2.26080E-05 \\
					8 & 9.16780E-05 \\
					9 & 3.55400E-04 \\
					10 & 1.31440E-03 \\
					11 & 4.62720E-03 \\
					12 & 1.54690E-02 \\
					13 & 4.89790E-02 \\
					14 & 1.46460E-01 \\
					15 & 4.12340E-01 \\
					16 & 1.08910E+00 \\
					17 & 2.68880E+00 \\
					18 & 6.17900E+00 \\
					19 & 1.31600E+01 \\
					20 & 2.58520E+01 \\
					21 & 4.66200E+01 \\
					22 & 7.68280E+01 \\
					23 & 1.15280E+02 \\
					24 & 1.57220E+02 \\
					25 & 1.95350E+02 \\
					26 & 2.23080E+02 \\
					27 & 2.38200E+02 \\
					28 & 2.43700E+02 \\
					29 & 2.44700E+02 \\
					30 & 2.44700E+02 \\
					\end{tabular}
			\end{table}
			\begin{list}
				\item For $k=29$, $e^{5.5}$ converges to 5-significant figures
				\item The value of $e^{5.5}$ computed using built-in function is $244.691932$
				\item The relative error is $0.000033$
			\end{list}
	\item
\end{enumerate}

\newpage
\section{Recurrence in reverse}
\begin{enumerate}[(a)]
	\item The reverse recurrence relation is given by:
			\begin{align*}
				y_{n-1} = \frac{e - y_{n}}{n}
			\end{align*}
		Computing for a few terms down the chain we obtain:
			\begin{align*}
				y_{n-2} &= 	\frac{e - y_{n-1}}{n-1}\\
						&= 	\frac{ne - e + y_n}{n\br{n-1}}\\
				y_{n-3}	&=	\frac{e - y_{n-2}}{n-2}\\
						&=	\frac{n\br{n-1}e -ne + e- y_n}{n\br{n-1}\br{n-2}}
			\end{align*}
		One can denote the pattern as:
			\begin{align*}
				y_{n-p} = (-1)^{p} \frac{y_n}{n\br{n-1}\br{n-2}\dots\br{n-p+1}} + e &\left[\frac{1}{n-\br{p-1}} - \frac{1}{\br{n-\br{p-1}}\br{n-\br{p-2}}}\right.\\
																					&+ \left.\frac{1}{\br{n-\br{p-1}}\br{n-\br{p-2}}\br{n-\br{p-3}}} + \dots \right]
			\end{align*}
		To obtain the value of $y_k$ in terms of $y_N$ we replace $n-p$ with $k$ and simplify:
			\begin{align*}
				y_k &= (-1)^{n-k} \frac{y_n}{n\br{n-1}\br{n-2}\dots\br{k+1}} + \mt{exponent \ terms}\\
					&=(-1)^{n-k} \frac{y_n k!}{n!} + \mt{exponent \ terms}
			\end{align*}
			The condition number is given as:
			\begin{align*}
				\br{\mt{cond} \ g_k}\br{y_k} 	&= \left|\frac{y_N g'\br{y_N}}{y_k}\right| \\
												&= \left|\frac{y_N \frac{k!}{N!}}{y_k}\right|
			\end{align*}
			Since the $k$ is less than $N$, $y_k$ is greater than $y_N$, the upper bound on the condition number, $\br{\mt{cond} \ g_k}\br{y_k}$, obtained as:
			\begin{align*}
				\br{\mt{cond} \ g_k}\br{y_k} \leq \frac{k!}{N!}
			\end{align*}
			as $\frac{y_N}{y_k} \leq 1$
	\item We know the condition number represents:
			\begin{align*}
				\varepsilon_y &= \br{\mt{cond} \ g_k} \varepsilon_x \\
				\frac{\Delta y_k}{y_k} &\leq \frac{k!}{N!} \quad \leq \varepsilon\\
				N! &\geq \frac{k!}{\varepsilon}
			\end{align*}
			Here, we have assumed $\varepsilon_x = 1$ and $\varepsilon$ is a predefined target error in $y_k$.
	\item For \py{python3} the machine epsilon for float is $1.0e^{-15}$(Obtained using numpy.finfo(float)). Using this machine epsilon the value of $N$ obtained is 31. [Check code]
	\item The computed value of $y_{20}$ is $0.123803830762570$ and the value of $y_{20}$ directly by integration is $0.123803830762570$. [Check code]
\end{enumerate}

%
%\section{Error propagation in exponentiation}
%\begin{enumerate}[(a)]
%	\item Derive the upper bound on relative error resulting from machine arithmetic for the two different algorithms
%		\begin{enumerate}[(i)]
%			\item Multiplying x's one we get:
%				\begin{align*}
%					\mt{fl}\left(x \times x\right) = x^2 (1+\epsilon)
%				\end{align*}
%				By induction, the relative error in computing $x^n$ by repeated multiplication is $\left(1+\epsilon\right)^{n-1}$. If we neglect any terms of type $\Oc\left(\mt{eps}^2\right)$ and higher the error can be written as:
%				\begin{align*}
%				\mt{Relative \ Error} = \br{n-1}\epsilon
%				\end{align*}
%			
%			\item Using exponentiation:
%				\begin{align*}
%					\mt{fl}\br{e^{\mt{fl}\br{n\br{\mt{fl}\br{\mt{ln}x}}}}}	&= \mt{fl}\br{e^{\mt{fl}\br{n\br{\mt{ln}x\br{1+\epsilon_l}}}}}\\
%																			&= \mt{fl}\br{e^{n \mt{ln}x \br{1+\epsilon_l+\epsilon_n}}}\\
%																			&= e^{n \mt{ln}x \br{1+\epsilon_l+\epsilon_n}}\br{1+\epsilon_{\mt{exp}}}\\
%																			&= x^{n\br{1+\epsilon_l+\epsilon_n}}\br{1+\epsilon_{\mt{exp}}}\\
%																			&= x^n\br{x^{n\br{\epsilon_l+\epsilon_n}}\br{1+\epsilon_{\mt{exp}}}}
%				\end{align*}
%		\end{enumerate}
%	\item Suppose $x$ is positive and $a$ is nonzero. Determine the propagated relative error $\epsilon$ in $x^a$ when:
%		\begin{enumerate}[(i)]
%			\item $x$ is an exact machine number  but $a$ is subject to a relative error $\epsilon_a$
%				
%			\item $a$ is an exact machine number  but $x$ is subject to a relative error $\epsilon_x$
%		\end{enumerate}
%		NOTE: Since $a$ is an arbitrary positive number we only focus on the exponentiation method
%\end{enumerate}
%
%\section{Conditioning}
%Consider the function
%	\begin{align*}
%		f(x) = 1 - e^{-x}
%	\end{align*}
%\begin{enumerate}[(a)]
%	\item The condition (cond $f$)($x$) of the function in terms of x is given as:\\
%			\begin{align*}
%			(\mt{cond \ } f)(x)	&= \left|x \frac{f'(x)}{f(x)}\right|\\
%								&= \left|x \frac{e^{-x}}{1-e^{-x}}\right|\\
%								&= \left|\frac{x}{e^x -1}\right|
%			\end{align*}
%		The function $\frac{x}{e^x -1}$ is a monotonically decreasing function between $\left[0,1\right]$ with values being $1$ at $0$ and $0.58$ at $1$. Hence the problem is well conditioned on this interval
%	\item 
%\end{enumerate}
%
%\section{Limits in $\Rbb\left(p,q\right)$}
%
%\begin{enumerate}[(a)]
%	\item The code stopped at $n = 10^{17}$
%	\item The final converged value is $1.0$
%	\item A table of the intermediate values computed for $0 \leq n \leq n_{\mt{stop}}$
%	\begin{table}[H]
%		\centering
%		\begin{tabular}{c|c}
%		n 	& value \\
%		\hline
%		1.0E+00 & 2.0000000000000 \\
%		1.0E+01 & 2.5937424601000 \\
%		1.0E+02 & 2.7048138294215 \\
%		1.0E+03 & 2.7169239322356 \\
%		1.0E+04 & 2.7181459268249 \\
%		1.0E+05 & 2.7182682371923 \\
%		1.0E+06 & 2.7182804690958 \\
%		1.0E+07 & 2.7182816941321 \\
%		1.0E+08 & 2.7182817983474 \\
%		1.0E+09 & 2.7182820520116 \\
%		1.0E+10 & 2.7182820532348 \\
%		1.0E+11 & 2.7182820533571 \\
%		1.0E+12 & 2.7185234960372 \\
%		1.0E+13 & 2.7161100340869 \\
%		1.0E+14 & 2.7161100340870 \\
%		1.0E+15 & 3.0350352065493 \\
%		1.0E+16 & 1.0000000000000 \\
%		1.0E+17 & 1.0000000000000 		
%		\end{tabular}
%	\end{table}
%\end{enumerate}
%The reason why it converges to that value is that the term $\frac{1}{n}$ gets ignored when n reaches $10^{16}$ as the machine epsilon for floating point in python3 is $\approx 10^{-17}$ (This can be tested by computing $0.1+0.1+0.1-0.3$).
%\section{Fun with square roots}
%\begin{figure}[H]
%	\begin{center}
%		\begin{tabular}{c c} %% tabular useful for creating an array of images 
%			\includegraphics[width=0.45\textwidth]{plot_for_n_49} & \includegraphics[width=0.45\textwidth]{plot_for_n_50} \\
%			\includegraphics[width=0.45\textwidth]{plot_for_n_51} & \includegraphics[width=0.45\textwidth]{plot_for_n_52}
%		\end{tabular}
%	\end{center}
%	\caption[funwithsquareroots] 
%	%>>>> use \label inside caption to get Fig. number with \ref{}
%	{ Plots for different values of $n$}
%\end{figure}
%\section{The issue with polynomial roots}
%\begin{enumerate}[(a)]
%	\item The coefficients of the polynomial are:
%		\begin{table}[H]
%			\centering
%			\begin{tabular}{r|r}
%			$n$ & $a_n$\\
%			\hline
%			0	&	2432902008176640000 \\		
%			1	&	-8752948036761600000 \\
%			2	&	13803759753640704000 \\
%			3	&	-12870931245150988800 \\
%			4	&	8037811822645051776 \\
%			5	&	-3599979517947607200 \\
%			6	&	1206647803780373360\\
%			7	&	-311333643161390640\\
%			8	&	63030812099294896\\
%			9	&	-10142299865511450\\
%			10	&	1307535010540395\\
%			11	&	-135585182899530\\
%			12	&	11310276995381\\
%			13	&	-756111184500\\
%			14	&	40171771630\\
%			15	&	-1672280820\\
%			16	&	53327946\\
%			17	&	-1256850\\
%			18	&	20615\\
%			19	&	-210\\
%			20	&	1
%			\end{tabular}
%		\end{table}
%	\item Yes, The newton raphson method converges to $20.00003189$ when the initial guess of $21$ is provided.\\
%			Using the inbuilt function the largest root obtained is $20.00054209$, which is pretty much the same.Further various complex roots are calculated using the inbuilt function
%	\item Changing the coefficient $a_{20}$ from $1 \rightarrow 1+\delta$
%		\begin{table}[H]
%			\centering
%			\begin{tabular}{c|c|c}
%				$\delta$ 	& Largest root using NR & Largest root using inbuilt function \\
%				\hline
%				$10^{-8}$	& $9.5854$ 	& 	$20.648+1.1869j$\\
%				$10^{-6}$	& $7.7527$	&	$23.149 +2.7410j$\\
%				$10^{-4}$	& $5.9693$	&	$28.400 +6.5104j$\\
%				$10^{-2}$	& $5.4696$	&	$38.478 +20.834j$
%			\end{tabular}
%		\end{table}
%	\item Changing the value of coefficient $a_{19}$ from $-210 \rightarrow -210-2^{-23}$
%	\item Consider a monic degree-$n$ polynomial
%\end{enumerate}



\end{document}


